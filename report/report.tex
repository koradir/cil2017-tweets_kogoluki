\documentclass[10pt,conference,compsocconf]{IEEEtran}

\usepackage[super]{nth}
\usepackage{url}
\usepackage{graphicx}	% For figure environment
\usepackage{xcolor}
\usepackage{hyperref} %keep hyperref at the bottom of the usepackage list

\newcommand{\todo}{\textcolor{red}{\textbf{!!TODO!!}}}
\newcommand{\todoThis}[1]{\textcolor{red}{\textbf{!!TODO: #1}}}

\begin{document}
\title{ETHZ CIL 2017 : Tweet Sentiment Analysis}

\author{
  Roger Koradi\\
  \texttt{koradir@ethz.ch}
  \and
  Luzi Sennhauser\\
  \texttt{luzis@ethz.ch}
  \and
  Flavio Goldener\\
  \texttt{gflavio@ethz.ch}
  \and
  Georg Kilzer\\
  \texttt{gkilzer@ethz.ch}
  %\\
  %Department of Computer Science, ETH Zurich, Switzerland
}

\maketitle

%\begin{figure}[htbp]
%  \centering
%  \includegraphics[width=\columnwidth]{local_wdenoised_1d}
%  \caption{Signal compression and denoising using the Daubechies wavelet basis.}
%  \label{fig:denoise-wavelet}
%\end{figure}

\begin{abstract}
\todoThis{The abstract should really be written last, along with the title of
the paper. The four points that should be covered:
\begin{enumerate}
\item State the problem.
\item Say why it is an interesting problem.
\item Say what your solution achieves.
\item Say what follows from your solution.
\end{enumerate}}
\end{abstract}

\section{Introduction}\label{sec:introduction}

\todoThis{Describe your problem and state your
  contributions.}

bib-test:\\  
  Glove paper: \cite{glove_paper}\\
  NLTK blog: \cite{nltk_blog}\\
  CNN paper: \cite{cnn_paper}\\
  CNN baseline: \cite{cnn_base}
  
\section{Models and Methods}\label{sec:models}
\todoThis{The model refers to the underlying mathematical model or structure which 
you use to describe your problem, or that your solution is based on. 
The methods on the other hand, are the algorithms used to solve the problem. \\\\Describe your idea and how it was implemented to solve
  the problem. Survey the related work, giving credit where credit is
  due.\\\\It is usually helpful to
structure the methods section:
\begin{enumerate}
\item Layout the model you used to describe the problem or the solution.
\item Describing the algorithms used in the study, briefly including
  details such as hyperparameter values (e.g. thresholds), and
  preprocessing steps (e.g. normalizing the data to have mean value of
  zero).
\item Explaining how the materials were prepared, for example the
  images used and their resolution.
\item Describing the research protocol, for example which examples
  were used for estimating the parameters (training) and which were
  used for computing performance.
\item Explaining how measurements were made and what
  calculations were performed. Do not reproduce the full source code in
  the paper, but explain the key steps.
\end{enumerate}
}

\section{Results}\label{sec:results}
\todoThis{Show evidence to support your claims made in the
  introduction.}
  
\section{Discussion}\label{sec:discussion}
\todoThis{Discuss the strengths and weaknesses of your
  approach, based on the results. Point out the implications of your
  novel idea on the application concerned.}
  
\todoThis{You compare your novel algorithm to \emph{at least two baseline
  algorithms}. For the baselines, you can use the implementations you
developed as part of the programming assignments.\\}
  
\section{Summary}\label{sec:summary}
\todoThis{Summarize your contributions in light of the new
  results.}

\section*{Acknowledgements}
\todo

or remove section if none

\bibliographystyle{IEEEtran}
\bibliography{report}
\section{(DELETE before submission)}
\todoThis{Remove this section before submission}\\

Your semester project is a group effort. It consists of four parts:
\begin{enumerate}
\item The programming assignments you solve during the semester.
\item Developing a novel solution for one of the assignments, e.g. by
  combining methods from previous programming assignments into a novel
  solution.
\item Comparing your novel solution to previous assignments.
\item Writing up your findings in a short scientific paper.
\end{enumerate}

There are two different types of grading criteria applied to your
project, with the corresponding weights shown in brackets.
\begin{description}
\item[Competitive] \ \\
  The following criteria is scored based on your rank
  in comparison with the rest of the class.
  \begin{itemize}
  \item time taken for computation (10\%)
  \item average rank for all other criteria relevant to the task, for
    example reconstruction error and sparsity (20\%)
  \end{itemize}
  The ranks will then be converted on a linear scale into a grade
  between 4 and 6.
\item[Non-competitive] \ \\
  The following criteria is scored based on an
  evaluation by the teaching assistants.
  \begin{itemize}
  \item quality of paper (30\%)
  \item quality of implementation (20\%)
  \item creativity of solution (20\%)
  \end{itemize}
\end{description}

\textcolor{red}{The document should be a maximum of {\bf 4 pages}.}\\

As for code...
\begin{itemize}
\item Have a \texttt{README} file that (at least) describes what your
  software does, and which commands to run to obtain results. Also
  mention anything special that needs to be set up, such as
  toolboxes\footnote{For those who are
  particularly interested, other common structures can be found at
  \url{http://en.wikipedia.org/wiki/README} and
  \url{http://www.gnu.org/software/womb/gnits/}.}.
\item A list of authors and contributors can be included in a file
  called \texttt{AUTHORS}, acknowledging any help that you may have
  obtained. For small projects, this information is often also
  included in the \texttt{README}.
\item Use meaningful filenames, and not \texttt{temp1.m},
  \texttt{temp2.m}. The code should also unzip into a subdirectory.
\item Document your code. Each file should at least have a short
  description about its reason for existence. Non obvious steps in the
  code should be commented.
\item Describe how the results presented in your paper can potentially
  be reproduced.
\end{itemize}
\end{document}
